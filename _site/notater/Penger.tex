%&latex
\documentclass[a4paper,notitlepage]{article}
\usepackage{beamerarticle}
%\mode<article>{\usepackage{fullpage}}
%\documentclass[notes=show]{beamer}
%%%%%%%%%%%%%%%%%%%%%%%%%%%%%%%%%%%%%%%%%%%%%%%%%%%%%%%%%%%%%%%%%%%%%%%%%%%%%%%%%%%%%%%%%%%%%%%%%%%%%%%%%%%%%%%%%%%%%%%%%%%%%%%%%%%%%%%%%%%%%%%%%%%%%%%%%%%%%%%%%%%%%%%%%%%%%%%%%%%%%%%%%%%%%%%%%%%%%%%%%%%%%%%%%%%%%%%%%%%%%%%%%%%%%%%%%%%%%%%%%%%%%%%%%%%%
\usepackage{amssymb}
\usepackage{mathpazo}
\usepackage{hyperref}
\usepackage{multimedia}
\usepackage{graphicx}
\usepackage{amssymb}
%\usepackage[english]{babel}
\usepackage[latin1]{inputenc}
%\usepackage{amsmath}
\usetheme{Madrid}
%\input{tcilatex}
\usepackage{color}

\begin{document}

\title[Triola kapittel 1]{Penger, kreditt og bankadferd}
\author[JIH]{J\o rn I. Halvorsen \\ Universitet i ?s}
\institute[BI]{\inst{1} }
\date[Sj\o krigsskolen 2007]{Forelesningsnotat ECN 222, h?sten 2013}
\maketitle
%------------------------------
\section{Introduksjon}
I en moderne ?konomi er dype nedgangskonjunkturer omtrent alltid koblet sammen med grunnleggende problemer i den finansielle sektoren i ?konomien.  Siden kredittlinjene som utg?r fra finanssektoren ber?rer omtrent samtlige av ?konomiens akt?rer, vil en kollaps av det finansielle systemet ogs? medf?re et stort fall i produksjonen som en f?lge av ?kte kostnader tilknyttet det ? gjennomf?re ?konomiske transaksjoner.

P? bakgrunn av et slikt scenario, vil offentlige myndigheter vanligvis ha interesse av ? spille en aktiv rolle n?r det gjelder ? minimere skadene ved en finansiell kollaps og forhindre slike kriser fra ? skje p? nytt.
Men det eksisterer forel?pig ikke noen historiske erfaringer som kan gi oss noen klar pekepinn p? hvordan disse problemene b?r l?ses. Videre gir heller ikke konjunkturteoriene som vi til n? har l?rt i dette kurset noen s?rlig innsikt i dette problemet.

Form?let med notatet er ? belyse dette problemomr?det n?rmere ved ? gi en grov oversikt over de viktigste problemene som offentlige myndigheter m?te med utformingen av det finansielle systemet samt ogs? gi et innblikk i hvordan ulike akt?rer kan forvente ? tjene eller tape p? disse l?sningene.
M?ten vi g?r fram p?, er ? fors?ke gi klare svar p? fire sp?rsm?l som er knyttet til utformingen av det finansielle systemet.
\begin{itemize}
\item Hva er penger?
\item Hvilken funksjoner og utforminger har bankvesenet for et land typisk hatt?
\item Hvorfor trenger vi en sentralbank?
\item Hvorfor kan banker ha ?nske om ? p?dra seg un?dvendig systemrisiko?
\end{itemize}
Avhengig av hvordan en ?konomi er organisert, kan penger spille alt fra en ubetydelig til en betydelig rolle i en ?konomi.  Dessuten kunne ha helt forskjellige utforminger.

\begin{itemize}
\item Simpel bytte?konomi\\
For ? oppn? varebytte, vil akt?rene inng? en avtale om ? bytte det de selv eier med andre akt?rer i ?konomien.\\
\noindent\textbf{\emph{\\ \\ Eksempel}}\\\
Agent 1 bytter varer med agent 2 ved ? motta vare A i bytte for vare B\\
\noindent\textbf{\\ Problemer:}\\
(1)Krever at akt?rene har behov for hverandres varer. \\
(2)Manglende delbarhet. \\
(3)Regneenhet.
\item Bytte?konomi med indirekte bytte\\
En m?te ? l?se det f?rste problemet p?, er ? bytte til seg en vare som en selv ikke trenger (indirekte bytte) for deretter benytte denne varen som byttemiddel mot andre varer.\\
\noindent\textbf{\emph{\\ Eksempel}}\\\
Agent 1 bytter varer med agent 3 ved ? motta vare A i bytte for vare C. Vare C benyttes deretter av agent 1 for ? bytte vare C mot vare B med agent 2.
\item ?konomi med penger som har en verdi i seg selv\\
Dersom ?konomien best?r av en rekke forskjellige akt?rer og varer, viser historien at en uregulert ?konomi vil f?re at akt?rene klarer selv ? koordinere seg fram til et lite utvalg av varer som fungerer som byttemiddel i omtrent alle ?konomiske transaksjoner. Det er dette selekterte utvalget av varer som vi skal benevne som penger som har verdi i seg selv. Disse varene har typisk egenskaper i form av:
\begin{enumerate}
\item Delbare.
\item B?rbare.
\item Lang varighet.
\item High verdi per vektenhet.
\end{enumerate}
Historisk s? er det gull og s?lv som har v?rt de dominerende varene innenfor penger som har verd i seg selv. Men i mer isolerte ?konomier, finnes ogs? eksempler som sigaretter, juveler og kameler. At gull og s?lv har hatt en slik dominerende rolle, er nok et resultat av at de er stand til effektivt l?se alle de problemene som ble nevnt ved en bytte?konomi.
 \item ?konomi med penger som gjeld\\
I dagens moderne ?konomier, har ikke penger lengre noen reell forankring til en vare som har en verdi i seg selv; men dreier seg om gjeldsbaserte penger utstedt av staten men ogs? banker med kredittskapende evner. Vi kan konkretisere dette litt n?rmere.
\begin{itemize}
\item Utvendige penger (fra staten):\\
Sedler, mynt pluss reserver som forretningsbankene har p? konto i sentralbanken.
\item Innvendige penge (fra bankene):\\
Gjeld skapt gjennom innskuddskontorer som kommer til syne p? forretningsbankenes balanse.
\end{itemize}
\end{itemize}
I offentlige tilgjengelige statistikker, vil beholdningen av utvendige penger i en ?konomi g? under betegnelsen $M0$ (eller M1), mens innvendige penger framkommer ved $M1-M0$ (eller M1-M2).\footnote{H?yere pengeaggregater enn M1 (M2) kan ogs? benyttes, men vi ser bort fra dette i det her kurset.}
Overgangen fra en ?konomi med penger som har verdi i seg selv til en ?konomi med penger som gjeld, har typisk skjedd gradvis over tid og kommet som en konsekvens av ulike utforminger av det finansielle systemet. Vi skal n? f?lge en slik utvikling ved ? se n?rmere p? tre arktyper av dette systemet, og vurdere hvordan dette er relatert til banksystemet i en ?konomi.
\section{Bankvesentets funksjoner og utforminger}
For en ?konomi uten banker og med penger som har verdi i seg selv, vil penger prim?rt benyttes til
\begin{itemize}
\item L?n og sparing.
\item Byttemiddel ved ?konomiske transaksjoner.
\end{itemize}
\noindent\textbf{\\ Problemer:}\\
(1) Vanskelig for l?ntakere og finne l?ngivere (pga. mangelfull informasjon).\\
(2) H?ye kostnader ved ? oppbevare penger. \\
\subsection{Banksystemets funksjoner}
Banksystemets hovedfunksjoner g?r utp? ? l?se disse de to problemene, ved ? tilby spesialkompetanse for vurdering av utl?n og betalingssystemer for bruk ved ?konomiske transaksjoner. F?r vi g?r videre med ? se p? hvordan banksystemet kan utformes, kan det kanskje f?rst v?re nyttig ? kjenne til tre ulike typer av banker.
\begin{itemize}
\item \textbf{Utl?nsbank}.\\
Den bankens funksjon er ? ivareta kundenes behov  for  ? l?ne og spare.\\

\noindent\textbf{\emph{Eksempel}}\\
\begin{tabular}{rlrr}
    \textbf{Eiendeler} & \textbf{} & \textbf{Gjeld og egenkapital} & \textbf{} \\
\hline    L?n   & 50    & 25    & Egenkapital \\
          &       & 25    & Obligasjoner \\
\hline          & 50    & 50    &  \\
\end{tabular}%
\end{itemize}
\begin{itemize}
\item \textbf{Transaksjonsbank}\\
Den bankens funksjon er ? ivareta behovet for transaksjonstjenester.\\

\noindent\textbf{\emph{Eksempel}}\

\begin{tabular}{rlrr}
 \textbf{Eiendeler} & \textbf{} & \textbf{Gjeld og egenkapital} & \textbf{}\\
\hline    Gullreserver &  150   & 150   & Innskudd \\
\hline                 & 150   & 150   &  \\
\end{tabular}
\end{itemize}
\begin{itemize}
\item \textbf{Forretningsbank} \\
Denne banken ivaretar funksjonen til b?de utl?nsbanken og forretningsbanken

\noindent \\\textbf{\emph{Eksempel}}\\
\begin{tabular}{rlrr}
    \textbf{Eiendeler} & \textbf{} & \textbf{Gjeld og egenkapital} & \textbf{} \\
\hline    L?n   & 50    & 25    & Egenkapital \\
    Gullreserver  &  150   & 25    & Obligasjoner \\
          &       & 150   & Innskudd \\
\hline          & 200   & 200   &  \\
\end{tabular}%
\end{itemize}
\subsection{System 1: L?nebasert bank?konomi}
I dette systemet blir bankenes utl?n finansier meg egenkapital og gjeld, mens eiereandelen til gullreservene endres i takt med de ?konomiske transaksjonene som foreg?r i ?konomien  \\

\noindent\textbf{\emph{Eksempel}}\\
\textbf{Forretningsbank (konsolidert)}
\begin{center}\begin{tabular}{rlrr}
    \textbf{Eiendeler} & \textbf{} & \textbf{Gjeld og egenkapital} & \textbf{} \\
\hline    L?n   & 50    & 25    & Egenkapital \\
    Gullreserver  &  150   & 25    & Obligasjoner \\
          &       & 150   & Innskudd \\
\hline          & 200   & 200   &  \\
\end{tabular}%
\end{center}
\noindent \textbf{Forretningsbank 1}
\begin{center}
\begin{tabular}{rrrr}
    \textbf{Eiendeler} & \textbf{} & \textbf{Gjeld og egenkapital} & \textbf{} \\
\hline    L?n   & 25    & 25    & Egenkapital \\
    Gullreserver & 75    & 75    & Innskudd klient 1 \\
\hline          & 100    & 100    &  \\
\end{tabular}%
\end{center}
\noindent \textbf{Forretningsbank 2}
\begin{center}

\begin{tabular}{rrrr}
    \textbf{Eiendeler} & \textbf{} & \textbf{Gjeld og egenkapital} & \textbf{} \\
\hline    L?n   & 25    & 25    & Obligasjoner \\
    Gullreserver & 75    & 75    & Innskudd klient 2 \\
\hline          & 100    & 100    &  \\
\end{tabular}%
\end{center}
\noindent\textbf{\emph{Eksempel}}\\
Forretningsbankenes balanse etter at klient 1 har kj?pt en vare av klient 2 for et bel?p p? 25
\noindent \textbf{Forretningsbank 1}
\begin{center}
\begin{tabular}{rrrr}
    \textbf{Eiendeler} & \textbf{} & \textbf{Gjeld og egenkapital} & \textbf{} \\
\hline    L?n   & 25    & 25    & Egenkapital \\
    Gullreserver & 75    & 50    & Innskudd klient 1 \\
\hline          & 100    & 100    &  \\
\end{tabular}%
\end{center}

\noindent \textbf{\\ \\ \\ \\ \\ Forretningsbank 2}
\begin{center}

\begin{tabular}{rrrr}
    \textbf{Eiendeler} & \textbf{} & \textbf{Gjeld og egenkapital} & \textbf{} \\
\hline    L?n   & 25    & 25    & Obligasjoner \\
    Gullreserver & 75    & 100    & Innskudd klient 2 \\
\hline          & 100    & 100    &  \\
\end{tabular}%
\end{center}
\noindent\textbf{Problemer:}\\
\begin{itemize}
\item Kostnader forbundet med ? flytte reserver mellom konkurerrende forretningsbanker.
\item Uelastisk kredittilbud (nye l?n kan f?rst bli tilgjengelig ved ?kt sparing)
\end{itemize}
\subsection{System 2: Fraksjonsbasert bank?konomi uten sentralbank}
Denne utformingen l?ser problemet med det uelastisk kreditttilbudet under system 1, ved at bankene kan utstede mer penger (banksedler eller kontopenger) enn det de har som reserver. Men det gj?r ogs? banksystemet mer ustabilt, spesielt dersom utl?nene er av d?rlig kvalitet\\
\noindent\textbf{\emph{\\ Eksempel}}\\

\textbf{Forretningsbank (konsolidert)}
\begin{center}\begin{tabular}{rlrr}
    \textbf{Eiendeler} & \textbf{} & \textbf{Gjeld og egenkapital} & \textbf{} \\
\hline    L?n   & \textcolor{red}{200}    & 25    & Egenkapital \\
    Gullreserver  &  150   & 25    & Obligasjoner \\
          &       & \textcolor{red}{300}    & Innskudd \\
\hline          & 350   & 350   &  \\
\end{tabular}%
\end{center}
\noindent \textbf{Forretningsbank 1}
\begin{center}
\begin{tabular}{rrrr}
    \textbf{Eiendeler} & \textbf{} & \textbf{Gjeld og egenkapital} & \textbf{} \\
\hline    L?n   & \textcolor{red}{100}    & 25    & Egenkapital \\
    Gullreserver & 50    & 50    & Innskudd klient 1 \\
  &     & \textcolor{red}{75}    & Innskudd klient 3 \\

\hline          & 75    & 75    &  \\
\end{tabular}%
\end{center}
\noindent \textbf{Forretningsbank 2}
\begin{center}

\begin{tabular}{rrrr}
    \textbf{Eiendeler} & \textbf{} & \textbf{Gjeld og egenkapital} & \textbf{} \\
\hline    L?n   &\textcolor{red}{100}   & 25    & Obligasjoner \\
    Gullreserver & 50    & 50    & Innskudd klient 1 \\
    Gullreserver & 50    & \textcolor{red}{75}    & Innskudd klient 1 \\
\hline          & 75    & 75    &  \\
\end{tabular}%
\end{center}
Beholdning ettter at b?de forretningsbank 1 og forretningsbank 2 har ?kt utl?n og innvendige penger i systemet med 75.
\bigskip\\

\noindent\textbf{\\ Problemer:}\\
\begin{itemize}
\item Fortsatt kostnader forbundet med ? flytte reserver mellom forretningsbankene.
\item ?pner for l?p p? bankene (bank har ikke nok gullreserver dersom alle bankens klienter samtidig ?nsker ? trekke ut sine gullreserver)          \end{itemize}
\subsection{System 3: Fraksjonsbasert bank?konomi med sentralbank}
Ved opprettelsen av en sentralbank vil kostnader ved ? flytte reserver mellom forretningsbanker bli sterkt redusert, fordi reservene blir sentralisert til en plass. Videre vil sentralbanken ogs? ha mulighet til tre st?ttende til i det det oppst?r en fare for en kollaps av den finansielle sektoren i ?konomien. Sentralbanken har trolig blitt etablert som et resultat av at flere sterke interessegrupper har hatt sterke egeninteresser av det:
 \begin{itemize}
\item Banksektoren kan i st?rre til ? forsikre seg mot l?p p? bankene.
\item Husholdningen oppn?r st?rre sikkerhet for sine innskudd.
\item Myndigheten kan n? finansiere sine budsjettunderskudd ved ? trykke penger (monetarisering av gjeld).
\end{itemize}
Sammenliknet med system3, vil bankene under dette systemet ikke lengre ha mulighet til ? utstede egne sedler. Den retten gis ene og alene til myndighetene, som n? har monopol. Man bankene vil fortsatte kunne ha mulighet til ? tilf?re systemet likviditet ved ? skape innvendige penger i form av innskuddskontoer og l?n.\footnote{(Egen)kapitalkrav og reservekrav er to instrumenter som myndigheten kan benytte til ? redusere banksektorene kredittskapende evner.}\\
\noindent\textbf{\emph{\\ Eksempel}}\\

\textbf{Forretningsbank (konsolidert)}
\begin{center}\begin{tabular}{rlrr}
    \textbf{Eiendeler} & \textbf{} & \textbf{Gjeld og egenkapital} & \textbf{} \\
\hline    L?n   & {200}    & 25    & Egenkapital \\
    Reserver  &  150+Y   & 25    & Obligasjoner \\
          &       & {300}    & Innskudd \\
          &       & \textcolor{red}{Y}    & L?n SB \\
\hline          & 350+Y   & 350+Y   &  \\
\end{tabular}%
\end{center}
\noindent \textbf{Forretningsbank 1}
\begin{center}
\begin{tabular}{rrrr}
    \textbf{Eiendeler} & \textbf{} & \textbf{Gjeld og egenkapital} & \textbf{} \\
\hline    L?n   & \textcolor{red}{100}    & 25    & Egenkapital \\
    Reserver & 75+Y1+Z    & 50    & Innskudd klient 1 og 3 \\
  &     & \textcolor{red}{Z}    & L?n bank 2(interbank) \\
  &     & \textcolor{red}{Y1}    & L?n SB \\

\hline          & 75+Y1+Z    & 75+Y1+Z    &  \\
\end{tabular}%
\end{center}

\noindent \textbf{Forretningsbank 2}
\begin{center}
\begin{tabular}{rrrr}
    \textbf{Eiendeler} & \textbf{} & \textbf{Gjeld og egenkapital} & \textbf{} \\
\hline    L?n   & \textcolor{red}{100}    & 25    & Obligasjoner \\
    Reserver & 75+Y2    & 50    & Innskudd klient 1 og 3 \\
  &     & \textcolor{red}{-Z}    & L?n bank 2(interbank) \\
  &     & \textcolor{red}{Y2}    & L?n SB \\

\hline          & 75+Y1-Z    & 75+Y1-Z    &  \\
\end{tabular}%
\end{center}
\noindent \textbf{Sentralbank}
\begin{center}

\end{center}
\begin{tabular}{rrrr}
    \textbf{Eiendeler} & \textbf{} & \textbf{Gjeld og egenkapital} & \textbf{} \\
\hline    Gull   &\textcolor{red}{200}   & 50+X+T    & Sedler og mynt \\
    Statsobligasjoner & X    & 75+Y1+Z           & Reserver bank 1 \\
    Andre papirer  & T    & \textcolor{red}{75+Y2-Z}    & Reserver bank 2 \\
    L?n til banker  & Y=Y1+Y2    &     &  \\

\hline          & 200+X+T+Y    & 200+X+T+Y    &  \\
\end{tabular}%
\bigskip\\
\bigskip
\bigskip\\

$Z$ representerr har bel?pet for internbankl?n mellom de to forretningsbankene. $X$ og $T$ er kj?p av statsobligasjoner og andre eiendeler (kvantitative lettelser) som vi antar her er finansiert ved fysisk pengetrykking (?kt beholdning av sedler og mynt). $Y=Y1+Y2$ representerer det aggregart l?nebel?pet som sentralbanken har gitt til forretnigsbankene (refinansieringsl?n).
\bigskip\\

\noindent Innenfor dette systemet, har det ogs? v?rt utpr?vd forskjellige varianter: \bigskip\\
\noindent\textbf{Sentralbank med reservekrav og innl?sningsmuligheter}\\
Myndigheten beholder her pengenes innl?sningsmulighet mot gull, men har mulighet til ? stille reservekrav til bankene for ? redusere sannsynligheten for at dette skal skje.\\

\noindent\textbf{Problemer:}\\
\begin{itemize}
\item For lavt reservekrav vil her f?re til l?p p? \emph{alle} bankene (systemkrise) i ?konomien.
\end{itemize}

\noindent\textbf{Sentralbank med reservekrav/kapitalkrav og offentlig innskudsgaranti}\\
\noindent I dag gir ikke lengre pengene som sentralbanken utsteder innl?sningsmuligheter mot gull, men bevarer til tross for dette sin kj?pekraft.  ?rsaken til dette er knyttet til at myndigheten gjennom lov har vedtatt at disse pengene er det gyldige betalingsmiddel for betaling av skatter avgifter. Frykten som innskyterne har for at de kan miste pengen de har p? konto i bankene er i sterk grad blitt redusert gjennom at myndighetene har garantert for innskyternes penger gjennom en offentlig innskuddsgaranti.\bigskip\\
\noindent\textbf{\\Problemer:}\\
\begin{itemize}
\item ?pner for at bankene tar un?dvendig stor risiko.
\item Kan f?re til en vilk?rlig og stor omfordeling av formue.
\end{itemize}
Vi skal n? dvele litt n?rmere ved de to siste problemene, ved ? studere i detalj et talleksempel som g?r under navnet ''The BLOOS rule''.
\section{Banker og systemrisiko}
\begin{quotation}
In Minsky's theory of endogenous financial booms and busts, prosperous
times lead to an excess availability of credit and the gradual development of credit-fuelled asset-price bubbles.  Financial crises result when those debt levels become excessive a so-called ''Minsky'' moment.  A credit crunch then ensues leading to a downturn in the real economy.
\end{quotation}
I dette avsnittet skal vi fors?ke ? skissere en sv?rt enkel modell med banker som kanskje er egnet til ? st?tte opp under dette sitatet.
\subsection{?konomien}
\textbf{Bedriftene}\\
Bedriftene i ?konomien er avhengig av finansiering fra bankene for ? kunne operere. De kan velge mellom to litt forskjellige forretningsmodeller:
\bigskip\\
\noindent Forretningsmodell 1:\\
\begin{center}
\begin{tabular}{lrr}
\hline
Avkastning     &  Tilstand & Sannsynlighet \\
\hline
$0,05$ &          1 & 0,99 \\
$0,05$ &          2 & 0,01 \\
\end{tabular}
\end{center}
Den forventede avkastningen ved forretningsmodell 1 (FM1) vil derfor v?re gitt ved $E(\text{FM1}_{1})=0,05 0,99+0,05\cdot 0,01=0,05$
\bigskip\\
Forretningsmodell 2:\\
\begin{center}
\begin{tabular}{lrr}
\hline
Avkastning     &  Tilstand & Sannsynlighet \\
\hline
$0,06$ &          1 & 0,99 \\
$-1$ &          2 & 0,01 \\
\end{tabular}
\end{center}
Den forventede avkastningen ved forretningsmodell 2 (FM2) vil v?re gitt ved $E(\text{FM2}_{1})=0,06\cdot 0,99-1\cdot 0,01=0,05$\bigskip\\

Som vi ser, har de to forretningsmodellene tiln?rmet lik forventningsverdi. Forretningsmodell 1 er imidlertid uten usikkerhet, mens forretningsmodell 2 inneholder risiko.\bigskip\\
\textbf{Husholdningene}\\
Husholdningene i ?konomien har valget mellom ? plassere pengene sine i sikre statsobligasjoner til en gitt rente, $r_{f}$, eller i bankobligasjoner i de to bankene gitt ved $r_{b1},r_{b2}$. Vi antar videre at husholdningen er risikon?ytrale, noe som gir lik forventet avkastning mellom statsobligasjoner og bankobligasjoner.\\

\noindent\textbf{Banksystemet}\\
Vi har to banker i ?konomien. Vi tenker oss at bank 1 finansierer forretningsmodell 1 og bank 2 finansierer forretningsmodell 2. Bankene finansierer seg selv gjennom egenkapital og gjeld i form av bankobligasjoner.\\

\noindent\textbf{Sentralbanken}\\
Sentralbanken bestemmer rente p? statsobligasjoner i ?konomien som er lik den risikofri avkastnibgnen i ?konomien. I v?rt eksempel setter vi denne lik 5\%.\\

\noindent\textbf{Myndighetene}\\
Mulighet til ? regulere egenkapitalkravniv?et som bankene m? innfri.\\

Basert p? opplysingnene gitt til n? og et investeringsbel?p p? 200 fordel likt mellom de to bankene, vil bedriftene og bankene kunne forvente f?lgende inntektsfordeling:
\begin{center}
\begin{tabular}{lrrr}
\hline
      & Forventet verdi & Tilstand 1 & Tilstand 2 \\
\hline
$\text{Bedrift 1}$ &       105 & 105 & 105 \\
$\text{Bedrift 2}$ &       105 & 106 & 0 \\
$\text{Bank 1}$    &       105 & 105 & 105 \\
$\text{Bank 2}$    &       105 & 106 & 0 \\

\end{tabular}
\end{center}
Hvor stor andel av disse inntektene som tilfaller aksjon?rene, obligasjonseierne og hvor store overf?ringer banksektoren mottar vil p?virkes av egenkapitalkravet og myndighetenes reguleringer. La oss se n?rmere p? noen spesialtilfeller.
\subsection{Systemrisiko og l?nnsomhet}
\subsubsection{Case 1: Ingen bailout guarantee og 10 prosent egenkapital}
For en husholdning vil dette bety at obligasjonsrenten for de to bankene vil v?re gitt ved
\begin{equation*}
(1+i_{rf})=1\cdot (1+i_{b_{1}})
\end{equation*}
Som gir $i_{b_{1}}=0,05$
\begin{equation*}
(1+i_{rf})=0,99\cdot (1+i_{b_{2}})+0,01\cdot 0=0,99\cdot (1+i_{b_{2}})
\end{equation*}
Som gir $i_{b_{1}}=0,0606$

\begin{center}
\begin{tabular}{llrrr}
\hline
      & Avkastning& Forventet verdi & Tilstand 1 & Tilstand 2 \\
\hline
$\text{Obligasjonseier bank 1 }$  &  &   94,50, & 94,50 & 94,50 \\
$\text{Obligasjonseier bank 2}$ &        & 95,45 & 95,45 & 0 \\
$\text{Aksjon?rer bank 1}$    &  5 \% &     10,50 & 10,50 & 10,50 \\
$\text{Aksjon?rer bank 2}$    &  5 \% &     10,50 & 10,61 & 0 \\

\end{tabular}
\end{center}

\subsubsection{Case 2: 100 prosent bailout guarantee}
\begin{equation*}
(1+i_{rf})=1\cdot (1+i_{b_{1}})=1\cdot (1+i_{b_{2}})
\end{equation*}
Som gir $i_{b_{1}}=0,05$
Vi antar ogs? her at egenkapitalen i bankene er p? 10 prosent og at sentralbanken setter renta p? statsobligasjoner til 5 prosent.
\begin{center}
\begin{tabular}{llrrr}
\hline
      & Avkastning& Forventet verdi & Tilstand 1 & Tilstand 2 \\
\hline
$\text{Obligasjonseier bank 1}$  & &   94,50, & 94,50 & 94,50 \\
$\text{Obligasjonseier bank 2}$ &        & 95,45 & 94,50 & 94,5\footnote{Merk: Dette bel?pet representere offentlige overf?ringer til bankene.} \\
$\text{Aksjon?rer bank 1}$    &  5 \% &     10,50 & 10,50 & 10,50 \\
$\text{Aksjon?rer bank 2}$    &  14,5 \% &     10,50 & 11,561 & 0 \\
\end{tabular}
\end{center}
Reduserer vi egenkapitalkravet til bankene til 5 prosent f?r vi
\begin{center}
\begin{tabular}{llrrr}
\hline
      & Avkastning& Forventet verdi & Tilstand 1 & Tilstand 2 \\
\hline
$\text{Obligasjonseier bank 1}$  & &   99,75, & 99,75 & 99,75 \\
$\text{Obligasjonseier bank 2}$ &        & 99,75 & 99,75 & 99,75\footnote{Dette bel?pet representere statlige overf?ringer.} \\
$\text{Aksjon?rer bank 1}$    &  5 \% &     5,25 & 5,25 & 5,25 \\
$\text{Aksjon?rer bank 2}$    &  25 \% &     6,25 & 6,3106 & 0 \\
\end{tabular}
\end{center}
Gitt at vi har konkurranse  i banksektoren, vil det v?re press p? bank 1 ? oppf?rer seg som bank 2 siden denne bankene har h?yere forventet avkastning. Da ender vi rask opp en l?sning hvor hvor begge bankene g?r inn for ? velge forretningsmodell 2, med den konsekvens av at ?konomien som helhet utsettes for en systemkrise  i det tilstand 2 inntreffer.
\subsubsection{Hva er hensikten bak et krav om h?yere egenkapitalandel for bankene?}
?nsket fra myndigheten om h?yere egenkapitalandel til bankene er knyttet spesielt til to forhold.
\begin{itemize}
\item (1) Lavere forventede utbetalinger fra myndighetene ved systemrisiko
\item (2) Substitusjon vekk fra forretningsmodeller med overdreven risiko
\end{itemize}
I dette talleksemplet vil vi ikke f? noen substitusjon fra forretningsmodell 1 til forretningsmodell 2, siden forretningsmodell 2 alltid vil ha en h?yere forventet avkastning. Men som vi kan se, vil den forventede utbetalingene i tilstand 2 g? ned siden bank i st?rre grad m? dekke tapene. I talleksemplet som ble gitt i ?velsesoppgaven, med en avkastning i tilstand 1 p? 0,056 for forretningsmodell 2, vil imidlertid en  substitusjon fra forretningsmodell 2 skje for en egenkapitalandel p? rundt 55 prosent
\end{document}
